\documentclass[a4paper,11pt]{article}
\usepackage[utf8]{inputenc}
\usepackage[T1]{fontenc}
\usepackage[french]{babel}
\usepackage{graphicx}
\usepackage{hyperref}
\usepackage{geometry}
\usepackage{listings}
\usepackage{xcolor}
\usepackage{tikz}
\usetikzlibrary{shapes.geometric, arrows, positioning, fit, backgrounds, shadows, calc}

% Configuration de la page
\geometry{hmargin=2.5cm,vmargin=2.5cm}

% Configuration des hyperliens
\hypersetup{
    colorlinks=true,
    linkcolor=blue,
    filecolor=magenta,      
    urlcolor=cyan,
}

% Configuration des listings de code
\definecolor{codegreen}{rgb}{0,0.6,0}
\definecolor{codegray}{rgb}{0.5,0.5,0.5}
\definecolor{codepurple}{rgb}{0.58,0,0.82}
\definecolor{backcolour}{rgb}{0.95,0.95,0.92}

\lstdefinestyle{mystyle}{
    backgroundcolor=\color{backcolour},   
    commentstyle=\color{codegreen},
    keywordstyle=\color{magenta},
    numberstyle=\tiny\color{codegray},
    stringstyle=\color{codepurple},
    basicstyle=\ttfamily\footnotesize,
    breakatwhitespace=false,         
    breaklines=true,                 
    captionpos=b,                    
    keepspaces=true,                 
    numbers=left,                    
    numbersep=5pt,                  
    showspaces=false,                
    showstringspaces=false,
    showtabs=false,                  
    tabsize=2
}
\lstset{style=mystyle}

% En-tête et pied de page standard
\pagestyle{plain}

\title{
    \vspace{2cm}
    \textbf{\huge Analyse et Exploitation de la CVE-2025-55182 (React2Shell)} \\
    \vspace{1cm}
    \large Sécurité des Systèmes d'Information
}
\author{Antonio Mattar, Matheo Dupiat, Romain Stablo}
\date{15 Janvier 2026}

\begin{document}

\maketitle
\thispagestyle{empty}
\newpage

\tableofcontents
\newpage

\section{Description de la Vulnérabilité}

\subsection{Présentation}
La CVE-2025-55182, surnommée \textbf{"React2Shell"}, est une vulnérabilité critique d'exécution de code à distance (RCE) affectant les versions récentes de React (19.x) et Next.js (15.x/16.x) utilisant l'architecture "App Router". Elle a obtenu un score CVSS de \textbf{10.0} (Critique).

\subsection{Origine et Fonctionnement}
La faille réside dans le mécanisme de désérialisation du protocole \textbf{React Server Components (RSC) Flight}.

\begin{itemize}
    \item \textbf{Contexte :} Next.js permet aux clients d'envoyer des objets sérialisés au serveur (pour les Server Actions par exemple).
    \item \textbf{Défaut :} Le désérialiseur serveur ne vérifie pas correctement les types des objets reçus, permettant une "Pollution de Prototype" ou l'injection d'objets malveillants lors de la reconstruction de l'arbre de composants.
    \item \textbf{Exploitation :} Un attaquant peut envoyer un payload JSON spécialement conçu qui, une fois désérialisé, incite le serveur à :
    \begin{enumerate}
        \item Charger des modules arbitraires (via `module.require`).
        \item Exécuter des commandes système (via `child\_process.execSync`).
    \end{enumerate}
\end{itemize}

Ce processus se déroule \textbf{avant} toute authentification applicative, rendant tout serveur exposé immédiatement vulnérable.

\newpage
\section{Architecture et Analyse}

\subsection{Architecture Typique Vulnérable}
L'architecture cible est une application web moderne basée sur Next.js, souvent déployée via Docker ou Kubernetes. Le schéma ci-dessous illustre l'infrastructure réseau et le flux de l'attaque.

\begin{center}
\begin{tikzpicture}[node distance=1.5cm and 1cm, auto, >=latex', thick]
    % Styles
    \tikzstyle{block} = [rectangle, draw, fill=blue!10, text width=2.5cm, text centered, rounded corners, minimum height=1.5cm, drop shadow]
    \tikzstyle{actor} = [circle, draw, fill=red!20, text width=1.5cm, text centered, minimum height=1.5cm, drop shadow]
    \tikzstyle{container} = [rectangle, draw, dashed, fill=gray!5, inner sep=0.8cm, rounded corners, label=above:\textbf{Serveur Docker / Kubernetes}]
    
    % Nodes
    \node[actor] (attacker) {Attaquant};
    
    % Internal components positioned relatively
    \node[block, right=2.5cm of attacker, fill=green!10] (nginx) {Reverse Proxy\\(Nginx)};
    \node[block, right=1.5cm of nginx, fill=blue!15] (nextjs) {Next.js App\\(App Router)};
    \node[block, right=1.5cm of nextjs, fill=yellow!10] (fs) {Système de\\Fichiers};
    
    % Container background
    \begin{pgfonlayer}{background}
        \node[container, fit=(nginx) (nextjs) (fs)] (box) {};
    \end{pgfonlayer}
    
    % Arrows with better labelling spacing
    \draw[->, red, very thick] (attacker) -- node[above, font=\footnotesize, text width=2cm, align=center] {1. Payload RSC\\(Malveillant)} (nginx);
    \draw[->, red, very thick] (nginx) -- node[above, font=\footnotesize] {2. Forward} (nextjs);
    \draw[->, red, very thick] (nextjs) -- node[above, font=\footnotesize] {3. RCE} (fs);
    
    % Return paths curved to avoid overlap
    \draw[->, blue, dashed] (nextjs.south) to[bend left=45] node[below, font=\footnotesize] {4. Output} (nginx.south);
    \draw[->, blue, dashed] (nginx.south) to[bend left=45] node[below, font=\footnotesize] {5. Réponse HTTP} (attacker.south);
    
    % Vulnerability marker
    \node[above=0.1cm of nextjs, text=red!80!black, font=\bfseries\small] {Vulnérabilité ici};
\end{tikzpicture}
\end{center}

\subsection{Flux d'Exploitation (Détail de la Vulnérabilité)}
Le schéma suivant détaille le processus interne de la désérialisation défaillante.

\begin{center}
\begin{tikzpicture}[node distance=1.5cm, auto, >=stealth]
    % Styles
    \tikzstyle{process} = [rectangle, draw, fill=orange!10, text width=4cm, text centered, minimum height=1cm, rounded corners]
    \tikzstyle{decision} = [diamond, draw, fill=blue!10, text width=2.5cm, text centered, inner sep=0pt]
    \tikzstyle{line} = [draw, -latex', thick]
    
    \node[process] (input) {Réception Payload JSON\\(Client RSC)};
    \node[process, below=of input] (deser) {Désérialisation React\\(Server Component)};
    \node[decision, below=of deser] (check) {Vérification Types?};
    \node[process, right=1.5cm of check, fill=red!20] (eval) {Reconstruction Objet +\\Exécution Code (RCE)};
    \node[process, below=1.5cm of check, fill=green!20] (safe) {Rendu Composant Sûr};
    
    % Paths
    \path[line] (input) -- (deser);
    \path[line] (deser) -- (check);
    \path[line] (check) -- node[left] {OUI (Patché)} (safe);
    \path[line] (check) -- node[above] {NON (Vulnérable)} (eval);
    \path[line, dashed, red] (eval) |- (safe.east);
    
    % Context info
    \node[right=0.5cm of input, text width=4cm, font=\footnotesize, align=left] {En-tête: \texttt{Next-Action}\\Contenu: \texttt{module.require}};

\end{tikzpicture}
\end{center}

\subsection{Détails de l'Infrastructure}
\begin{itemize}
    \item \textbf{Machines concernées :} Le serveur hébergeant l'application Node.js.
    \item \textbf{Services associés :} 
    \begin{itemize}
        \item \textbf{Next.js Server :} Gère le rendu (SSR) et les requêtes API. C'est ici que la désérialisation a lieu.
        \item \textbf{Reverse Proxy (Optionnel) :} Peut ne pas filtrer les requêtes RSC car elles ressemblent à du trafic légitime.
    \end{itemize}
    \item \textbf{Réseau :} L'application est généralement exposée sur le port 3000 (interne) ou 80/443 (externe).
\end{itemize}

\newpage
\section{Analyse de la Cible de Sécurité}

Une analyse formelle selon la définition de Cible de Sécurité (Security Target).

\subsection{Biens (Assets) à Protéger}
\begin{enumerate}
    \item \textbf{Intégrité du Serveur :} Le système d'exploitation hôte ou le conteneur ne doit pas être modifié par des tiers non autorisés.
    \item \textbf{Confidentialité des Données :} Fichiers de configuration (\texttt{.env}, clés API), bases de données, code source.
    \item \textbf{Disponibilité du Service :} L'application doit rester accessible aux utilisateurs légitimes (risque de Déni de Service via RCE).
\end{enumerate}

\subsection{Menaces (Threats)}
\begin{itemize}
    \item \textbf{T.RCE (Remote Code Execution) :} Un attaquant exécute des commandes arbitraires avec les privilèges du processus Node.js.
    \item \textbf{T.EXFIL (Exfiltration) :} Vol de données sensibles (clés AWS, mots de passe BDD) via lecture de fichiers ou variables d'environnement.
    \item \textbf{T.LATERAL (Mouvement Latéral) :} Utilisation du serveur compromis comme pivot pour attaquer le réseau interne.
\end{itemize}

\subsection{Fonctions de Sécurité (Security Functions)}
\begin{itemize}
    \item \textbf{SF.DESERIALIZATION (Déficiente) :} Le mécanisme de désérialisation RSC doit valider strictement les types d'objets entrants. \textit{(C'est la fonction défaillante ici)}.
    \item \textbf{SF.ISOLATION :} Utilisation de conteneurs (Docker) et d'utilisateurs non-privilégiés pour limiter l'impact d'une compromission.
    \item \textbf{SF.FILTERING :} WAF pour bloquer les payloads suspects contenant des chaînes comme `child_process` ou `__proto__`.
\end{itemize}

\newpage
\section{Contre-mesures Administrateur}

En tant qu'administrateur système ou DevOps, voici les dispositions à prendre pour limiter ou éviter les conséquences.

\subsection{Actions Correctives (Patching)}
La mesure la plus efficace est la mise à jour des composants affectés.
\begin{itemize}
    \item \textbf{Next.js :} Mettre à jour vers la version \textbf{16.0.7} ou supérieure (ou 15.2.4+).
    \item \textbf{React :} Utiliser les versions patchées \textbf{19.0.1}, \textbf{19.1.2} ou \textbf{19.2.1}.
\end{itemize}

\subsection{Limitation de l'Impact (Défense en Profondeur)}
\begin{enumerate}
    \item \textbf{Principe de Moindre Privilège :}
    \begin{itemize}
        \item Ne jamais faire tourner le processus Node.js en \texttt{root}.
        \item Créer un utilisateur dédié (ex: `node`) dans le Dockerfile.
    \end{itemize}
    \item \textbf{Système de Fichiers en Lecture Seule :}
    \begin{itemize}
        \item Monter le rootfs en \texttt{read-only} pour empêcher l'installation de malwares ou backdoors persistantes.
    \end{itemize}
    \item \textbf{Segmentation Réseau :}
    \begin{itemize}
        \item Isoler le conteneur applicatif dans un VLAN ou un réseau Docker restreint.
        \item Bloquer les connexions sortantes (egress filtering) pour empêcher les reverse shells.
    \end{itemize}
\end{enumerate}

\subsection{Surveillance et Filtrage}
\begin{itemize}
    \item \textbf{WAF :} Configurer des règles pour bloquer les requêtes contenant des signatures d'exploitation RSC (ex: en-tête `Next-Action` suspect couplé à des mots clés JSON malveillants).
    \item \textbf{Logging :} Surveiller les journaux pour détecter des processus enfants inattendus (`sh`, `bash`, `curl`) lancés par Node.js.
\end{itemize}

\newpage
\section{Expérimentation}

Nous avons mis en place un laboratoire complet pour reproduire cette faille.

\subsection{Mise en Oeuvre du Laboratoire}
L'environnement utilise Docker Compose pour orchestrer :
\begin{itemize}
    \item \textbf{vulnerable-app :} Un conteneur Next.js 16.0.6 (vulnérable).
    \item \textbf{exploit-client :} Une machine attaquante avec les scripts Python.
\end{itemize}

\begin{lstlisting}[language=bash, caption=Démarrage du Lab]
# Démarrage via le script fourni
./lab.sh start

# Vérification
curl -s http://localhost:3000
\end{lstlisting}

\subsection{Scénario d'Exploitation}

\subsubsection{Objectif}
Lire le fichier secret \texttt{/flag.txt} situé sur le serveur ("Capture the Flag"), puis soumettre le résultat dans l'interface web prévue à cet effet.

\subsubsection{Exécution de l'Attaque}
Nous utilisons un script Python (`exploit.py`) qui forge une requête HTTP POST multipart.

\begin{lstlisting}[language=bash, caption=Commande d'exploitation]
python3 exploit/exploit.py http://localhost:3000 "cat /flag.txt"
\end{lstlisting}

\subsubsection{Résultat}
Le serveur exécute la commande et renvoie le résultat dans la réponse d'erreur RSC.

\begin{verbatim}
[*] Target: http://localhost:3000
[*] Command: cat /flag.txt
[*] Sending exploit payload...

[+] VULNERABLE! Command output:
ENSIMAG{R34CT_S3RV3R_C0MP0N3NTS_RCE}
\end{verbatim}

\subsubsection{Validation}
L'utilisateur copie ensuite ce flag (`ENSIMAG{...}`) dans le formulaire de l'application web pour valider le succès de l'opération.

\begin{center}
    \fbox{
        \begin{minipage}{0.8\textwidth}
            \centering
            \textbf{Illustration de l'Interface Web} \\
            \includegraphics[width=\textwidth]{images/experimentation.png}
        \end{minipage}
    }
\end{center}

\newpage
\section{Glossaire et Références}

\subsection{Glossaire}
\begin{itemize}
    \item \textbf{RSC (React Server Components)} : Composants React rendus exclusivement sur le serveur.
    \item \textbf{RCE (Remote Code Execution)} : Capacité d'un attaquant à exécuter du code arbitraire sur une machine distante.
    \item \textbf{SSR (Server-Side Rendering)} : Génération du HTML côté serveur.
    \item \textbf{CVE (Common Vulnerabilities and Exposures)} : Liste publique des failles de sécurité.
\end{itemize}

\subsection{Références}
\begin{enumerate}
    \item \textbf{NIST NVD}, "CVE-2025-55182 Detail", \url{https://nvd.nist.gov/vuln/detail/CVE-2025-55182}
    \item \textbf{Wiz Research}, "React2Shell: Pwning Next.js servers remotely", \url{https://www.wiz.io/blog/nextjs-cve-2025-55182-react2shell-deep-dive}
    \item \textbf{React Team}, "Security Advisory: RSC Payload Deserialization", \url{https://react.dev/security}
    \item \textbf{Projet GitHub du Lab}, \textit{Code source fourni avec ce rapport}.
\end{enumerate}

\end{document}
